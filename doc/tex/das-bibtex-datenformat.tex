\section{Das \mbox{\BibTeX-}Datenformat}
\label{sec:das-bibtex-datenformat}

\BibTeX{} ist ein System zur Beschreibung und Interpretation von
Quellenverzeichnissen.
Entwickelt wurde es 1985 von Oren Patashnik und Leslie Lamport für das
professionelle Textsatzsystem \TeX{} und dessen Makropaket \LaTeX, wird jedoch
heute auch außerhalb dieser Umgebung genutzt, um Literaturquellen zu
beschreiben.\cite{wiki:BibTeX}

\medskip
\noindent\BibTeX{} besteht aus zwei Komponenten:
\begin{description}
    \item[Datenbankdateien:] Alle Quellen, auf die sich eine Publikation
        stützt, werden in einer Datenbank untergebracht. Dabei handelt es sich
        um eine einfache Textdatei, in der jede Quelle mit ihrem Typ (z.B. Buch
        oder Zeitschrift) und ihren Eigenschaften (z.B. Autor oder Untertitel)
        beschrieben wird. Das \mbox{\BibTeX-}Format ist sowohl für Menschen als
        auch Maschinen gut lesbar.
    \item[Interpreter:] Datenbankdateien werden von einem Programm eingelesen
        und analysiert. Abhängig vom Verwendungszweck und den Einstellungen des
        Programms werden dann bestimmte Informationen aus der Datenbank
        formatiert in ein Dokument eingefügt. Zum Beispiel kann das
        \mbox{\LaTeX-}Paket \file{bibtex} ein Literaturverzeichnis erstellen,
        das lediglich diejenigen Quellen enthält, die auch im Dokument explizit
        zitiert wurden. Es erzeugt außerdem automatisch nach einem wählbaren
        Muster Kürzel für die Quellen an den Stellen, an denen sie zitiert
        werden.
\end{description}

\subsection{Allgemeines Datenschema einer \mbox{\BibTeX-}Datenbank}

\mbox{\BibTeX-}Datenbanken sind Listen von Datensätzen (vgl.
\autoref{fig:structure-BibTeXDatabase}). Diese Datensätze (im Folgenden als
\emph{Elemente} bezeichnet) lassen sich grob in drei Arten unterteilen:
\begin{itemize}
    \item Einträge, die die eigentlichen Quellen beschreiben
    \item String-Elemente, die der Definition von String-Konstanten dienen
    \item Präambel-Elemente, die zur Ausführung von Code verwendet werden
\end{itemize}
Diese Elementarten werden später in diesem Abschnitt noch genauer beschrieben.

\begin{figure}
    \centering
    \begin{tikzpicture}[with background]
    \begin{scope}
        [ semithick
        , every node/.style={anchor=base}
        ]
        \matrix {
                           \node[database element] (p)  {Präambel};
            & \node {,}; & \node[database element] (s1) {String};
            & \node {,}; & \node[database element] (e1) {Eintrag};
            & \node {,}; & \node[database element] (e2) {Eintrag};
            & \node {,}; & \node[database element] (s2) {String};
            & \node {,}; & \node[database element] (e3) {Eintrag};
            & \node {,}; & \node {\dots};
            \\
        };
        \graph[use existing nodes] {
            e1 ->[bend right] s1;
            e2 ->[bend left] s1;
            e1 ->[bend left] e3;
        };
    \end{scope}
    \end{tikzpicture}
    \caption{Struktur einer \mbox{\BibTeX-}Datenbank}
    \label{fig:structure-BibTeXDatabase}
\end{figure}

Manche Elemente einer \mbox{\BibTeX-}Datenbank können andere referenzieren.
Für eine korrekte Interpretation muss das \mbox{\BibTeX-}Programm die
betroffenen Elemente in einer bestimmten Reihenfolge aus der Datenbank lesen.
\mbox{\BibTeX-}Datenbanken besitzen in ihrer abstrakten Datenstruktur daher eine
Ordnung.

\subsubsection{Einträge}

Ein Eintrag beschreibt genau eine Quelle. Zwar darf andersherum theoretisch
dieselbe Quelle von mehreren Einträgen beschrieben werden, dies hätte aber keinen Mehrwert. Hingegen ist es gängige Praxis, dass mehrere Einträge unterschiedliche Quellen aus
\emph{derselben Publikation} beschreiben -- also zum Beispiel verschiedene
Abschnitte eines einzigen Fachbuchs.

\begin{figure}
    \centering
    \begin{tikzpicture}[with background]
    \begin{scope}
        [ tree layout
        , grow=down
        , semithick
        , every node/.style={anchor=base}
        ]
        \node[database element] {Eintrag}
            child {node {Typ}}
            child {node {ID}}
            child {node {Felder}};
    \end{scope}
    \end{tikzpicture}
    \caption{Struktur eines Eintrags}
    \label{fig:structure-entry}
\end{figure}

Jeder Eintrag besitzt die folgenden drei Eigenschaften (vgl.
\autoref{fig:structure-entry}):

\begin{description}
    \item[Typ:] Die Art der Quelle, also z.B. \enquote{Buch}, \enquote{Artikel}
        oder \enquote{Tagungsband}.
    \item[ID:] Eine eindeutige Bezeichnung, anhand derer der Eintrag
        referenzierbar ist. In \LaTeX{} kann man bei Verwendung des
        \mbox{\file{bibtex}-}Pakets zum Beispiel den in \autoref{lst:id-usage}
        gezeigten Code schreiben.
\begin{lstlisting}[language=TeX,label=lst:id-usage,caption=Beispiel für die Nutzung einer Eintrags-ID]
% [...]
wird jedoch heute auch außerhalb dieser Umgebung genutzt, um Literaturquellen zu
beschreiben.\cite{wiki:BibTeX}
\end{lstlisting}
        \raggedright{
            Der Befehl \lstinlineTeX|\cite{wiki:BibTeX}| erzeugt eine Referenz
            auf die Quelle mit der ID \mbox{\lstinlineplain|wiki:BibTeX|}. Das
            Ergebnis kann man in \autoref{sec:das-bibtex-datenformat} dieses
            Dokuments sehen.
        }
    \item[Felder:] Einfache Name-Wert-Paare zur näheren Beschreibung des
        Eintrags, wie zum Beispiel $(Autor,\text{Wilhelm Busch})$ oder
        $(Jahr,1978)$.
\end{description}

\subsubsection{Präambel-Elemente}

Die Präambel ist ein speziell für die Verwendung mit \TeX{} oder \LaTeX{}
gedachtes Feature von \BibTeX{}. Über sie kann man Code spezifizieren, der
vom \mbox{\BibTeX-}Programm ausgeführt werden soll, bevor die Bibliographie
aufgebaut wird. Beispielsweise könnte man so \mbox{\TeX-}Makros definieren, die
dann innerhalb der Datenbank aufrufbar sind.

\subsubsection{Zeichenketten}

\def\temp{\enquote{Wikipedia, Die freie Enzyklopädie}}

Innerhalb einer \mbox{\BibTeX-}Datenbank können Werte mehrfach an verschiedenen
Orten auftreten. Zwei typische Beispiele dafür sind die folgenden:
\begin{itemize}
    \item Oft ist Wikipedia eine sehr hilfreiche Quelle für einzelne spezielle
        Informationen, die sich nicht in Fachbüchern finden. Es ist daher
        sicherlich keine Seltenheit, dass innerhalb einer Arbeit mehrere
        Wikipedia-Seiten zitiert werden. Diese benötigen alle einen eigenen
        Eintrag in der Datenbank, aber denselben Enzyklopädie-Titel \temp.
    \item Aus einem Fachbuch werden verschiedene Stellen zitiert. Man legt dafür
        einzelne Einträge mit konkreten Seitenzahlen an, die aber in allen
        anderen Feldern (Autor, Titel, \dots) identisch sind.
\end{itemize}
Damit solche Wiederholungen nicht immer wieder neu vollständig ausgeschrieben
werden müssen, lassen sich Aliase für Zeichenketten\footnote{
    {\titlelike Zeichenkette:} Abfolge von Buchstaben, Ziffern und sonstigen Schriftzeichen
} festlegen. Beispielsweise definiert man, dass die Zeichenkette \temp{} den
Namen \file{wikiTitle} bekommen soll. Danach kann man in allen Werten, in denen der Titel der Wikipedia stehen soll, einfach den Alias \file{wikiTitle}
einfügen. Das \mbox{\BibTeX-}Programm ersetzt diesen dann bei der Interpretation
der Datenbank durch den eigentlichen Wert.

Konkret können \mbox{\BibTeX-}Datenbanken \emph{String-Elemente} enthalten. Ein
String-Element besteht aus einer oder mehreren Definitionen für Zeichenketten-Aliase (siehe \autoref{fig:structurestring}).

\begin{figure}
    \centering
    \begin{tikzpicture}[with background]
    \begin{scope}
        [ tree layout
        , grow=down
        , semithick
        , every node/.style={anchor=base}
        ]
        \node[database element] {String}
            child { node {Alias-Definition}
                    child {node {Name}}
                    child {node {Wert}}
                  }
            child { node {Alias-Definition}
                    child {node {Name}}
                    child {node {Wert}}
                  }
            child { node[minimum height=1.2em] {\dots}
                    edge from parent[child anchor=north]
                  };
    \end{scope}
    \end{tikzpicture}
    \caption{Struktur eines String-Elements}
    \label{fig:structurestring}
\end{figure}

\subsection{Überblick über die konkrete \mbox{\BibTeX-}Syntax}
