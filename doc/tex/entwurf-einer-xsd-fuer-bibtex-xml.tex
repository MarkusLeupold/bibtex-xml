\section[Entwurf einer XSD]{Entwurf einer XSD für \BibTeXXML}

Eine XML-Schema-Definition (XSD) eignet sich hervorragend, um das
\BibTeXXMLformat{} zu definieren. Es lässt sich sehr genau einschränken, welche
Werte an verschiedenen Stellen im Dokument erlaubt sind (z.B. für die ID eines
Eintrags oder den Namen eines Feldes). Auch kann im Gegensatz zur DTD definiert
werden, dass Eintrags-IDs eindeutig sein müssen.

Es wurde eine XSD im Venetian-Blind-Design nach der Beschreibung in
\cite{oracle:xsd-design-patterns} entworfen. Sie befindet sich im Abgabeordner
unter dem Dateinamen \file{BibTeX-XML.xsd}. Die Validierbarkeit des Dokuments
mit dieser XSD wurde mit dem Programm \file{xmllint} durch folgenden Aufruf
überprüft:
\begin{lstlisting}[language=bash]
$ xmllint --schema BibTeX-XML.xsd --noout Projekt_BIB_original.xml
Projekt_BIB_original.xml validates
\end{lstlisting}

Eine Version der erzeugten Datenbank mit XML-Schema-Instanz-Deklaration findet
sich in der Datei \file{Projekt\_BIB\_original\_xsd.xml}. Ob die
Instanz-Deklaration korrekt funktioniert konnte nicht überprüft werden, da
\file{xmllint} diese nicht automatisiert verarbeiten kann.