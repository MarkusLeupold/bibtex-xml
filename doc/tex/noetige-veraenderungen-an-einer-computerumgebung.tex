\section[Nötige Veränderungen an einer Computerumgebung]{
    Nötige Veränderung an einer Computerumgebung zum Testen des Arbeitsergebnisses
}

Grundsätzlich muss hier erwähnt werden, dass das Projekt auf einem Linux-System
(Ubuntu) bearbeitet wurde. Die in dieser Dokumentation gezeigten
Kommandozeilenaufrufe sind in \file{bash}-Syntax geschrieben und können daher
nicht identisch auf einem Windows-System verwendet werden. Der Autor kann keine
Angaben über spezifische Vorgehensweisen auf einem anderen Betriebssystem
machen.

Im Folgenden werden die nötigen Veränderungen genannt, die Linux-Systeme
betreffen.
\begin{description}
    \item[Installation von \file{bibtoxml}:] Falls der Bedarf besteht, die
        Konvertierung von \BibTeX{} zu \BibTeXXML{} zu testen, so kann dafür
        das GitHub-Repository dieses Projekts (siehe \cite{github-bibtex-xml})
        geklont werden, in dem der Quellcode des Konvertierungsprogramms
        enthalten ist. Zum Kompilieren des Programms benötigt man den
        Haskell-Compiler \file{GHC} und optional das Programm \file{make}.
    \item[Installation von \file{xmllint}:] \file{xmllint} wurde in diesem
        Projekt für die Validierung von XML-Dokumenten nach DTDs oder XSDs
        verwendet. Das Programm ist für Linux verfügbar, soll aber laut
        \cite{xmlsoft-org} auch auf anderen Betriebssystem (z.B. Windows)
        kompilieren. Der Autor kann darüber keine genauen Angaben machen.
    \item[Installation von \file{xsltproc}:] \file{xsltproc} wurde in diesem
        Projekt verwendet, um ein XSLT-Stylesheet auf XML-Dateien anzuwenden.
        Auch dieses Programm sollte laut \cite{xmlsoft-org} auch auf anderen
        Betriebssystemen verwendbar sein.
\end{description}