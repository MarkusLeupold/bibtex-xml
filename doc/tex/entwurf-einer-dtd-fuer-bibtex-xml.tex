\section[Entwurf einer DTD]{Entwurf einer DTD für \BibTeXXML}

Die entwickelte XML-Struktur lässt sich nur sehr ungenau mittels einer XML-Dokumenttypdefinition (DTD) beschreiben. Im Folgenden sind die größten Schwierigkeiten aufgeführt:

\begin{description}
    \item[XML-Namensräume:] Da XML-Namensräume nicht Teil der
        XML-1.0-Spezifikation sind, gibt es in DTDs keine Unterstützung für
        diese. Man muss sich also entscheiden, ob man den Dokumenttyp mit oder
        ohne vollqualifizierte Namen (d.h. mit oder ohne Präfix) formuliert.
        
        Spezifiziert man ein Präfix, so muss dieses auch identisch im
        Instanzdokument verwendet werden -- es besteht also keine Wahlfreiheit
        bezüglich des Präfixes, wie sie eigentlich durch die Spezifikation
        \cite{w3c:xml-namespaces} vorgesehen ist.
        
        Nutzt man hingegen kein Präfix in der DTD, so darf auch im
        Instanzdokument kein Präfix verwendet werden.
        
        Die DTD für \BibTeXXML{} legt kein Präfix fest, aber spezifiziert auf den Datenbankelement ein
        Attribut \lstinlineXML|xmlns| mit festem Wert, der der URL des
        \mbox{\BibTeX-}XML-Namensraums entspricht. Dadurch wird ein Parser
        dieses Element zusammen mit seinen Kindelementen immer im
        Standardnamensraum \BibTeXXMLnamespace{} interpretieren.
    \item[Eintrags-IDs:] In \BibTeX{} dürfen Eintrags-IDs Zeichen enthalten, die
        in den DTD-Typen \lstinlineplain|ID| und \lstinlineplain|NMTOKEN| nicht
        erlaubt sind (z.B. \lstinlineplain|/|). Daher bleibt als einziger
        möglicher Typ für Eintrags-IDs der Typ \lstinlineplain|CDATA|, der zu
        viele Zeichen zulässt (z.B. \lstinlineplain|&|, \lstinlineplain|\|).
        Außerdem fällt dadurch die Einschränkung weg, dass die Eintrags-ID
        innerhalb einer Datenbank eindeutig sein muss.
\end{description}

Die entworfene DTD findet sich unter dem Dateinamen \file{BibTeX-XML.dtd} im
Abgabeordner. Die Datei \file{Projekt\_BIB\_original\_dtd.xml} enthält eine Version
der erzeugten Datenbank in der diese DTD verlinkt ist.

Die Validierbarkeit der Datenbank wurde mit dem Programm \file{xmllint} durch
folgenden Aufruf überprüft, wobei eine leere Ausgabe die Validierbarkeit bestätigt hat:
\begin{lstlisting}[language=bash]
$ xmllint --valid --noblanks --noout Projekt_BIB_original_dtd.xml
\end{lstlisting}
Das Argument \lstinlinebash|--noblanks| veranlasst \file{xmllint} dazu, beim
Einlesen der Datei allen vernachlässigbaren Leerraum zu entfernen. Ansonsten
werden die Zeilenumbrüche, die sich zur besseren Lesbarkeit im Dokument
befinden, als eigene Kindknoten interpretiert, die durch die DTD nicht erlaubt
sind.