\documentclass[a4paper,11pt]{scrartcl}
\usepackage[
    top=2cm,
    right=2cm,
    bottom=2cm,
    left=2.5cm,
    footskip=1cm
]{geometry}

\usepackage{xcolor}
\selectcolormodel{RGB}
\convertcolorsDtrue
\definecolor{text-darkred}{Hsb}{19,0.87,0.63}
\definecolor{text-blueishgray}{Hsb}{197,0.05,0.5}
\definecolor{background-lightgray}{Hsb}{0,0,0.90}
\definecolor{background-middlegray}{Hsb}{0,0,0.75}

\usepackage{hyperref}
\usepackage[ngerman]{babel}
\addto\extrasngerman{%
    \def\sectionautorefname{Abschnitt}%
    \def\subsectionautorefname{Unterabschnitt}%
    \def\subsubsectionautorefname{Unterunterabschnitt}%
    \def\figureautorefname{Abbildung}%
    \def\tableautorefname{Tabelle}%
}
\usepackage{myutil}

\usepackage{biblatex}
\addbibresource{bib/sources.bib}

\usepackage[autostyle]{csquotes}

\usepackage{tikz}
\usetikzlibrary
  { backgrounds
  , positioning
  , arrows.meta
  , calc
  , graphs
  , graphdrawing
  , shapes.geometric
  , decorations.pathmorphing
  , fit
  }
\usegdlibrary{trees}
\usegdlibrary{layered}

\def\clearrecordednodenames{\global\def\recordednodenames{}}
\clearrecordednodenames
\def\recordnodename#1{%
    \global\edef\recordednodenames{%
        \expandafter\noexpand\recordednodenames(#1)%
    }%
}
\tikzset
  { with background/.style=
      { show background rectangle
      , inner frame sep=0.75em
      , background rectangle/.style=
          { fill=background-lightgray
          , rounded corners=0.25em
          }
      }
  , record node names/.style={/tikz/name/.append code={\recordnodename{##1}}}
  , small rounded corners/.style={rounded corners=0.25ex}
  , rounded rectangle/.style={rectangle, draw, small rounded corners}
  , database element/.style={rounded rectangle}
  , xml element/.style={rounded rectangle, font=\ttfamily}
  , xml attribute/.style={ellipse, draw, font=\ttfamily}
  , operator/.style=
      { diamond
      , draw
      , small rounded corners
      , inner sep=0.2em
      , anchor=center
      , text depth=
      }
  , production/.style=
      { rectangle
      , draw
      , font=\ttfamily\itshape
      , decorate
      , decoration={random steps, segment length=0.22em, amplitude=0.04em}
      }
  }


\usepackage{listings}
\lstset{
    basicstyle=\ttfamily,
    numbers=left,numberstyle=\tiny,
    frame=single,aboveskip=\medskipamount,belowskip=\medskipamount,
    breaklines,
    basicstyle=\footnotesize\ttfamily,
    keywordstyle=\color{text-darkred},
    commentstyle=\itshape\color{text-blueishgray},
    caption=\lstname,
    moreemph={\cite,\LaTeX},
    emphstyle={\color{text-darkred}}
}
\def\lstinlinebasicstyle{\normalsize\ttfamily}
\def\lstinlineplain{\lstinline[basicstyle=\lstinlinebasicstyle]}
\def\lstinlineTeX{\lstinline[language=TeX,basicstyle=\lstinlinebasicstyle]}
\def\lstinlineXML{\lstinline[language=XML,basicstyle=\lstinlinebasicstyle]}
\def\lstinlinehtml{\lstinline[language=html,basicstyle=\lstinlinebasicstyle]}
\def\lstinlinebash{\lstinline[language=bash,basicstyle=\lstinlinebasicstyle]}

\def\BibTeXdatabase{\mbox{\BibTeX-}Datenbank}
\def\BibTeXXMLdatabase{\mbox{\BibTeX-}XML-Datenbank}
\def\BibTeXXMLnamespace{\url{https://github.com/MarkusLeupold/bibtex-xml}}
\def\BibTeXXML{\mbox{\BibTeX-}XML}
\def\BibTeXXMLformat{\mbox{\BibTeX-}XML-Format}
\def\BibTeXXMLdoc{\mbox{\BibTeX-}XML-Dokument}

\def\fBibTeXXMLtohtml{\file{BibTeX-XML-to-html.xsl}}
\def\fhtmlresult{\file{Projekt\_BIB\_original.html}}
\def\fxmlresultwithstylesheet{\file{Projekt\_BIB\_original\_with-style.xml}}


\title{Umwandlung einer \BibTeX-Datenbank in XML und HTML zur Darstellung
    mittels eines Webbrowsers}
\subtitle{Prüfungsprojekt Dokumentbeschreibungssprachen}
\def\authormail{markus.leupold@stud.htwk-leipzig.de}
\author{
    Markus~Leupold\\
    \href{mailto:\authormail}{{\ttfamily\authormail}}}


\begin{document}
    \maketitle
    \tableofcontents
    
    \section{Einführung in \mbox{\BibTeX}}
\label{sec:einfuehrung-in-bibtex}

\BibTeX{} ist ein System zur Beschreibung und Interpretation von
Quellenverzeichnissen.
Entwickelt wurde es 1985 von Oren Patashnik und Leslie Lamport für das
professionelle Textsatzsystem \TeX{} und dessen Makropaket \LaTeX, wird jedoch
heute auch außerhalb dieser Umgebung genutzt, um Literaturquellen zu
beschreiben.\cite{wiki:BibTeX}

\subsection{Die Kompenenten des \mbox{\BibTeX-}Systems}
\noindent\BibTeX{} besteht aus zwei Komponenten:
\begin{description}
    \item[Datenbankdateien:] Alle Quellen, auf die sich eine Publikation
        stützt, werden in einer Datenbank untergebracht. Dabei handelt es sich
        um eine einfache Textdatei, in der jede Quelle mit ihrem Typ (z.B. Buch
        oder Zeitschrift) und ihren Eigenschaften (z.B. Autor oder Untertitel)
        beschrieben wird. Das \mbox{\BibTeX-}Format ist sowohl für Menschen als
        auch Maschinen gut lesbar.
    \item[Interpreter:] Datenbankdateien werden von einem Programm eingelesen
        und analysiert. Abhängig vom Verwendungszweck und den Einstellungen des
        Programms werden dann bestimmte Informationen aus der Datenbank
        formatiert in ein Dokument eingefügt. Zum Beispiel kann das
        \mbox{\LaTeX-}Paket \file{bibtex} ein Literaturverzeichnis erstellen,
        das lediglich diejenigen Quellen enthält, die auch im Dokument explizit
        zitiert wurden. Es erzeugt außerdem automatisch nach einem wählbaren
        Muster Kürzel für die Quellen an den Stellen, an denen sie zitiert
        werden.
\end{description}

\subsection{Allgemeines Datenschema einer \mbox{\BibTeX-}Datenbank}

\mbox{\BibTeX-}Datenbanken sind Listen von Datensätzen (vgl.
\autoref{fig:structure-BibTeXDatabase}). Diese Datensätze (im Folgenden als
\emph{Elemente} bezeichnet) lassen sich grob in drei Arten unterteilen:
\begin{itemize}
    \item Einträge, die die eigentlichen Quellen beschreiben
    \item String-Elemente, die der Definition von String-Konstanten dienen
    \item Präambel-Elemente, die zur Ausführung von Code verwendet werden
\end{itemize}
Diese Elementarten werden später in diesem Abschnitt noch genauer beschrieben.

\begin{figure}
    \centering
    \begin{tikzpicture}[with background]
    \begin{scope}
        [ semithick
        , every node/.style={anchor=base}
        ]
        \matrix {
                           \node[database element] (p)  {Präambel};
            & \node {,}; & \node[database element] (s1) {String};
            & \node {,}; & \node[database element] (e1) {Eintrag};
            & \node {,}; & \node[database element] (e2) {Eintrag};
            & \node {,}; & \node[database element] (s2) {String};
            & \node {,}; & \node[database element] (e3) {Eintrag};
            & \node {,}; & \node {\dots};
            \\
        };
        \graph[use existing nodes] {
            e1 ->[bend right] s1;
            e2 ->[bend left] s1;
            e1 ->[bend left] e3;
        };
    \end{scope}
    \end{tikzpicture}
    \caption{Struktur einer \mbox{\BibTeX-}Datenbank}
    \label{fig:structure-BibTeXDatabase}
\end{figure}

Manche Elemente einer \mbox{\BibTeX-}Datenbank können andere referenzieren.
Für eine korrekte Interpretation muss das \mbox{\BibTeX-}Programm die
betroffenen Elemente in einer bestimmten Reihenfolge aus der Datenbank lesen.
\mbox{\BibTeX-}Datenbanken besitzen in ihrer abstrakten Datenstruktur daher eine
Ordnung.

\subsubsection{Einträge}

Ein Eintrag beschreibt genau eine Quelle. Zwar darf andersherum theoretisch
dieselbe Quelle von mehreren Einträgen beschrieben werden, dies hätte aber keinen Mehrwert. Hingegen ist es gängige Praxis, dass mehrere Einträge unterschiedliche Quellen aus
\emph{derselben Publikation} beschreiben -- also zum Beispiel verschiedene
Abschnitte eines einzigen Fachbuchs.

\begin{figure}
    \centering
    \begin{tikzpicture}[with background]
    \begin{scope}
        [ tree layout
        , grow=down
        , semithick
        , every node/.style={anchor=base}
        ]
        \node[database element] {Eintrag}
            child {node {Typ}}
            child {node {ID}}
            child {node {Felder}};
    \end{scope}
    \end{tikzpicture}
    \caption{Struktur eines Eintrags}
    \label{fig:structure-entry}
\end{figure}

Jeder Eintrag besitzt die folgenden drei Eigenschaften (vgl.
\autoref{fig:structure-entry}):

\begin{description}
    \item[Typ:] Die Art der Quelle, also z.B. \enquote{Buch}, \enquote{Artikel}
        oder \enquote{Tagungsband}.
    \item[ID:] Eine eindeutige Bezeichnung, anhand derer der Eintrag
        referenzierbar ist. In \LaTeX{} kann man bei Verwendung des
        \mbox{\file{bibtex}-}Pakets zum Beispiel den in \autoref{lst:id-usage}
        gezeigten Code schreiben.
\begin{lstlisting}[language=TeX,label=lst:id-usage,caption=Beispiel für die Nutzung einer Eintrags-ID]
% [...]
wird jedoch heute auch außerhalb dieser Umgebung genutzt, um Literaturquellen zu
beschreiben.\cite{wiki:BibTeX}
\end{lstlisting}
        \raggedright{
            Der Befehl \lstinlineTeX|\cite{wiki:BibTeX}| erzeugt eine Referenz
            auf die Quelle mit der ID \mbox{\lstinlineplain|wiki:BibTeX|}. Das
            Ergebnis kann man in \autoref{sec:einfuehrung-in-bibtex} dieses
            Dokuments sehen.
        }
    \item[Felder:] Einfache Name-Wert-Paare zur näheren Beschreibung des
        Eintrags, wie zum Beispiel $(Autor,\text{Wilhelm Busch})$ oder
        $(Jahr,1978)$.
\end{description}

\subsubsection{Präambel-Elemente}

Die Präambel ist ein speziell für die Verwendung mit \TeX{} oder \LaTeX{}
gedachtes Feature von \BibTeX{}. Über sie kann man Code spezifizieren, der
vom \mbox{\BibTeX-}Programm ausgeführt werden soll, bevor die Bibliographie
aufgebaut wird. Beispielsweise könnte man so \mbox{\TeX-}Makros definieren, die
dann innerhalb der Datenbank aufrufbar sind.

\subsubsection{Zeichenketten}

\def\temp{\enquote{Wikipedia, Die freie Enzyklopädie}}

Innerhalb einer \mbox{\BibTeX-}Datenbank können Werte mehrfach an verschiedenen
Orten auftreten. Zwei typische Beispiele dafür sind die folgenden:
\begin{itemize}
    \item Oft ist Wikipedia eine sehr hilfreiche Quelle für einzelne spezielle
        Informationen, die sich nicht in Fachbüchern finden. Es ist daher
        sicherlich keine Seltenheit, dass innerhalb einer Arbeit mehrere
        Wikipedia-Seiten zitiert werden. Diese benötigen alle einen eigenen
        Eintrag in der Datenbank, aber denselben Enzyklopädie-Titel \temp.
    \item Aus einem Fachbuch werden verschiedene Stellen zitiert. Man legt dafür
        einzelne Einträge mit konkreten Seitenzahlen an, die aber in allen
        anderen Feldern (Autor, Titel, \dots) identisch sind.
\end{itemize}
Damit solche Wiederholungen nicht immer wieder neu vollständig ausgeschrieben
werden müssen, lassen sich Aliase für Zeichenketten\footnote{
    {\titlelike Zeichenkette:} Abfolge von Buchstaben, Ziffern und sonstigen Schriftzeichen
} festlegen. Beispielsweise definiert man, dass die Zeichenkette \temp{} den
Namen \file{wikiTitle} bekommen soll. Danach kann man in allen Werten, in denen der Titel der Wikipedia stehen soll, einfach den Alias \file{wikiTitle}
einfügen. Das \mbox{\BibTeX-}Programm ersetzt diesen dann bei der Interpretation
der Datenbank durch den eigentlichen Wert.

Konkret können \mbox{\BibTeX-}Datenbanken \emph{String-Elemente} enthalten. Ein
String-Element besteht aus einer oder mehreren Definitionen für Zeichenketten-Aliase (siehe \autoref{fig:structurestring}).

\begin{figure}
    \centering
    \begin{tikzpicture}[with background]
    \begin{scope}
        [ tree layout
        , grow=down
        , semithick
        , every node/.style={anchor=base}
        ]
        \node[database element] {String}
            child { node {Alias-Definition}
                    child {node {Name}}
                    child {node {Wert}}
                  }
            child { node {Alias-Definition}
                    child {node {Name}}
                    child {node {Wert}}
                  }
            child { node[minimum height=1.2em] {\dots}
                    edge from parent[child anchor=north]
                  };
    \end{scope}
    \end{tikzpicture}
    \caption{Struktur eines String-Elements}
    \label{fig:structurestring}
\end{figure}

    \section[Entwurf einer XML-Struktur]{Entwurf einer XML-Struktur zur Repräsentation einer \mbox{\BibTeX-}Datenbank}

\def\temp{
    Die Präambel ist speziell für die Verwendung mit \TeX{} gedacht. Da \TeX{}
    eine turingvollständige Sprache ist, ist die Interpretation entsprechend
    aufwendig.
}

Die XML-Struktur zur Darstellung einer \mbox{\BibTeX-}Datenbank (im Folgenden
\enquote{\BibTeXXML}) soll im Rahmen dieser Arbeit möglichst genau den
Funktionsumfang des normalen \mbox{\BibTeX-}Formats übernehmen. Neben Einträgen
sollen deshalb auch Strings unterstützt werden. Präambel-Elemente vernachlässigt
die XML-Struktur, da deren Interpretation den Rahmen dieser Arbeit sprengen
würde\footnote{\temp} und dieses Feature wahrscheinlich auch nur sehr selten
Verwendung findet.

Alle XML-Elemente des \BibTeXXMLformat s gehören dem Namensraum mit der URL
\BibTeXXMLnamespace an. \autoref{fig:structure-bibtex-xml-document} gibt einen
Überblick über die Struktur eines \BibTeXXMLdoc s. In den folgenden
Unterabschnitten werden ausgewählte \mbox{\BibTeX-}XML-Elemente und
Designentscheidungen genauer erklärt.

\begin{figure}
    \centering
    \begin{tikzpicture}
    [ with background
    , semithick
    , every node/.style={anchor=base, text depth=0}
    , node distance=\baselineskip
    ]
    \begin{scope}
        [ tree layout
        , significant sep=1em
        , record node names
        ]

        \node[xml element, dashed] (xml root) {/}
            child
              { node[xml element] {BibTeX:database}
                child
                  { node[operator] {$*$}
                    child
                      { node[operator] {$|$}
                        child
                          { node[xml element] {BibTeX:entry}
                            child {node[xml attribute] {id}}
                            child {node[xml attribute] {type}}
                            child
                              { node[operator] {$*$}
                                child
                                  { node [xml element] {BibTeX:field}
                                    child {node[xml attribute] {name}}
                                    child {node[production] {valueProd\/}}
                                  }
                              }
                          }
                        child
                          { node [xml element] {BibTeX:string}
                            child {node[xml attribute] {name}}
                            child {node[production] {valueProd\/}}
                          }
                      }
                  }
              }
          ;
    \end{scope}
    \node[fit=\recordednodenames, rectangle] (main tree) {};

    \node
      (valueProd)
      [ production
      , matrix
      , matrix anchor=north
      , below=of main tree.south
      , label=
          { [production,decorate=false,anchor=north west,draw=none]
            north west:%
            valueProd:\/
          }
      ]
      { \begin{scope}[tree layout]
            \node[operator] {$+$}
                child
                  { node[operator] {$|$}
                    child {node[xml element] {BibTeX:literalValue}}
                    child {node[xml element] {BibTeX:referencedValue}}
                  }
              ;
        \end{scope}
        \\
      };
    \node
      (symbolKey)
      [ rectangle
      , draw
      , matrix
      , matrix anchor=north
      , below=of valueProd.south
      , column sep=1em
      , empty/.style={text height=0, inner sep=0, minimum size=2ex}
      ]
      { \node[xml element, empty, label={[anchor=west]east:XML-Element}] {};
      & \node
          [ production
          , empty
          , label=
              { [anchor=west]
                east:Produktion (Platzhalter für Baumbestandteil)
              }
          ]
          {}
          ;
      \\\node
          [ xml attribute
          , empty
          , label={[anchor=west]east:XML-Attribut}
          ]
          {}
          ;
      \\\node[operator, empty, label={[anchor=west]east:Operator}] {};
      \\
      };
\end{tikzpicture}

    \caption{Struktur eines \mbox{\BibTeX-}XML-Dokuments}
    \label{fig:structure-bibtex-xml-document}
\end{figure}

\subsection{Das Datenbankelement}

Das Wurzelelement einer \mbox{\BibTeX}-XML-Datenbank ist ein Element vom Typ
\mbox{\lstinlineXML|BibTeX:database|.} Dieses enthält -- analog zur Struktur
einer \mbox{\BibTeX-}Datenbank -- in beliebiger Abfolge Eintrags- und
String-Elemente. Da \mbox{\BibTeX-}Datenbanken keine Metainformationen besitzen,
erfordert das \mbox{\lstinlineXML|BibTeX:database|-}Element außer den genannten
Elementen keine Attribute oder sonstigen Kindelemente.

\subsection{Eintrags-Elemente}

Einträge werden durch den Elementtyp \lstinlineXML|BibTeX:entry| dargestellt.
Deren Struktur entspricht im Großen und Ganzen den Einträgen in einer normalen
\mbox{\BibTeX-}Datenbank. Für die einzelnen Felder gibt es den speziellen
Elementtyp \lstinlineXML|BibTeX:field|, von dem null oder mehr Kindelemente in
jedem Eintrag vorhanden sein können. Einträge dürfen also theoretisch leer sein,
wobei das interpretierende Programm dann einen Fehler ausgeben sollte. Diese
Entscheidung rührt daher, dass auch das \mbox{\file{bibtex}-}Paket in \LaTeX{}
nicht mit einem Fehler stoppt, wenn Pflichtfelder in einem Eintrag fehlen,
sondern nur Warnungen ausgibt.

Der Wert eines Felds hat die in \autoref{subsec:feld-und-stringwerte}
beschriebene Struktur.

\subsection{Stringdefinitionen}

Während in einer \mbox{\BibTeX-}Datenbank Alias-Definitionen immer in
String-Elemente eingefasst sind, werden in \BibTeXXML{} Aliase einzeln auf
Datenbanklevel definiert. Ein umgebendes String-Element nach dem Vorbild der
\mbox{\BibTeX-}Datenbank würde zu den eigentlichen Definitionen keine weiteren
Informationen ergänzen, sodass eine solche Schachtelung nur die Dokumentstruktur
unnötigerweise komplizierter machen würde. Jeder Alias wird durch ein Element
vom Typ \lstinlineXML|BibTeX:string| definiert. Der Wert eines einzelnen Strings
hat dieselbe Struktur wie der Wert eines Felds -- beschrieben in
\autoref{subsec:feld-und-stringwerte}.

\subsection{Feld- und Stringwerte}
\label{subsec:feld-und-stringwerte}

Felder und Strings können zusammengesetzte Werte haben. Jeder Bestandteil eines
Wertes ist dabei entweder ein literaler Wert oder eine Referenz auf einen
String. Daher lassen sich Feld- und Stringwerte durch eine beliebige Abfolge
mindestens eines Elements der folgenden zwei Typen darstellen:
\begin{description}
    \item[\ttfamily BibTeX:literalValue:] Ein literaler Wert. Jedes
        Element dieses Typs besitzt ein Kind vom XML-Typ \lstinlineXML|CDATA|,
        das exakt der Zeichenkette entspricht, die das Element repräsentiert.
        Der Wert eines \mbox{\lstinlineXML|BibTeX:literalValue|-}Elements ist
        damit genau dessen Inhalt.
    \item[\ttfamily BibTeX:referencedValue:] Eine Referenz auf einen
        String. Jedes Element dieses Typs ist leer aber besitzt ein
        \mbox{\lstinlineXML|name|-}Attribut, das den Namen des referenzierten
        Strings enthält. Der Wert eines
        \mbox{\lstinlineXML|BibTeX:referencedValue|-}Elements bestimmt sich wie
        folgt:
        \begin{enumerate}
            \item Sei \textit{r} das
                \mbox{\lstinlineXML|BibTeX:referencedValue|-}Element, dessen
                Wert bestimmt werden soll.
            \item Liste alle Elemente des vollständigen Dokumentbaums in
                Präorder auf. Das Resultat nennen wir die \textit{Liste aller
                Elemente}. Fakt: Diese Liste enthält \textit{r}.
            \item Suche in der Liste aller Elemente das letzte
                \mbox{\lstinlineXML|BibTeX:string|-}Element, das vor \textit{r}
                steht und dessen \mbox{\lstinlineXML|name|-}Attribut mit dem
                \mbox{\lstinlineXML|name|-}Attribut von \textit{r}
                übereinstimmt. Wenn dieses Element existiert, so ist der Wert
                von \textit{r} der Wert dieses Elements. Ansonsten behandle das
                Fehlen als einen Fehler (\textit{r} referenziert einen String,
                der nicht definiert ist).
        \end{enumerate}
\end{description}
Der Wert eines Felds oder Strings bestimmt sich durch Verkettung der Einzelwerte
seiner Kinder in Reihenfolge der Dokumentordnung.
    \section{Erzeugung des Datenstroms im \BibTeXXMLformat}

Für dieses Projekt wurden mehrere ähnliche \mbox{\BibTeX-}Datenbanken
vorgegeben, von denen eine gewählt und in das entwickelte XML-Format konvertiert
werden soll. Dieser Abschnitt erläutert die Auswahl der konkreten Datenbank und
erklärt, wie aus dieser ein \BibTeXXMLdoc{} generiert wird.

\subsection{Eigenes Konvertierungsprogramm}

\def\temp{%
    Einzige Einschränkung: Das Konvertierungsprogramm unterstützt keine
    Präambel-Elemente. Beim Versuch, eine Datenbank mit einem solchen Element zu
    konvertieren, endet das Programm mit einem Fehler.%
}

Der Autor hat sich dafür entschieden, selbst einen \mbox{\BibTeX-}Parser in
Haskell zu implementieren. Auf dieser Basis hat er dann ein
Konvertierungsprogramm geschrieben, das aus beliebigen
\mbox{\BibTeX-}Datenbanken\footnote{\temp} ein äquivalentes valides XML-Dokument
im zuvor beschriebenen \BibTeXXMLformat{} erzeugt. Das Programm findet sich im
GitHub-Repository dieses Projekts unter dem Namen \file{bibtoxml} (siehe
\cite{github-bibtex-xml}).

\subsection{Auswahl der Datenbank}

Sowohl \BibTeXXML{} als auch \file{bibtoxml} weisen keine Einschränkungen auf,
die für die Erfüllung der Aufgabenstellung von Bedeutung sind. Die vorgegebene
Maximal-Datenbank \file{Projekt\_BIB\_original.txt} lässt sich \emph{fehlerfrei
und ohne Ausschluss von Datensätzen} mit \file{bibtoxml} umwandeln und in
\BibTeXXML{} darstellen. Daher wird für das Projekt diese Maximal-Datenbank
verwendet -- im Folgenden bezeichnet als \enquote{die \BibTeXdatabase}.
Diese ist im Abgabeordner unter dem genannten Dateinamen enthalten.

\subsection{Umwandlung der Datenbank}

\begin{flushleft}
Der folgende Befehl auf einer Linux-Kommandozeile erzeugt aus der  \BibTeXdatabase{} das \BibTeXXMLdoc{} \file{Projekt\_BIB\_original.xml}:
\end{flushleft}
\begin{lstlisting}[language=bash]
$ bibtoxml -o Projekt_BIB_original.xml Projekt_BIB_original.txt
\end{lstlisting}
Das erzeugte Dokument befindet sich mit dem hier verwendeten Dateinamen im
Abgabeordner.

    \section[Entwurf einer DTD]{Entwurf einer DTD für \BibTeXXML}

Die entwickelte XML-Struktur lässt sich nur sehr ungenau mittels einer XML-Dokumenttypdefinition (DTD) beschreiben. Im Folgenden sind die größten Schwierigkeiten aufgeführt:

\begin{description}
    \item[XML-Namensräume:] Da XML-Namensräume nicht Teil der
        XML-1.0-Spezifikation sind, gibt es in DTDs keine Unterstützung für
        diese. Man muss sich also entscheiden, ob man den Dokumenttyp mit oder
        ohne vollqualifizierte Namen (d.h. mit oder ohne Präfix) formuliert.
        
        Spezifiziert man ein Präfix, so muss dieses auch identisch im
        Instanzdokument verwendet werden -- es besteht also keine Wahlfreiheit
        bezüglich des Präfixes, wie sie eigentlich durch die Spezifikation
        \cite{w3c:xml-namespaces} vorgesehen ist.
        
        Nutzt man hingegen kein Präfix in der DTD, so darf auch im
        Instanzdokument kein Präfix verwendet werden.
        
        Die DTD für \BibTeXXML{} legt kein Präfix fest, aber spezifiziert auf den Datenbankelement ein
        Attribut \lstinlineXML|xmlns| mit festem Wert, der der URL des
        \mbox{\BibTeX-}XML-Namensraums entspricht. Dadurch wird ein Parser
        dieses Element zusammen mit seinen Kindelementen immer im
        Standardnamensraum \BibTeXXMLnamespace{} interpretieren.
    \item[Eintrags-IDs:] In \BibTeX{} dürfen Eintrags-IDs Zeichen enthalten, die
        in den DTD-Typen \lstinlineplain|ID| und \lstinlineplain|NMTOKEN| nicht
        erlaubt sind (z.B. \lstinlineplain|/|). Daher bleibt als einziger
        möglicher Typ für Eintrags-IDs der Typ \lstinlineplain|CDATA|, der zu
        viele Zeichen zulässt (z.B. \lstinlineplain|&|, \lstinlineplain|\|).
        Außerdem fällt dadurch die Einschränkung weg, dass die Eintrags-ID
        innerhalb einer Datenbank eindeutig sein muss.
\end{description}

Die entworfene DTD findet sich unter dem Dateinamen \file{BibTeX-XML.dtd} im
Abgabeordner. Die Datei \file{Projekt\_BIB\_original\_dtd.xml} enthält eine Version
der erzeugten Datenbank in der diese DTD verlinkt ist.

Die Validierbarkeit der Datenbank wurde mit dem Programm \file{xmllint} durch
folgenden Aufruf überprüft, wobei eine leere Ausgabe die Validierbarkeit bestätigt hat:
\begin{lstlisting}[language=bash]
$ xmllint --valid --noblanks --noout Projekt_BIB_original_dtd.xml
\end{lstlisting}
Das Argument \lstinlinebash|--noblanks| veranlasst \file{xmllint} dazu, beim
Einlesen der Datei allen vernachlässigbaren Leerraum zu entfernen. Ansonsten
werden die Zeilenumbrüche, die sich zur besseren Lesbarkeit im Dokument
befinden, als eigene Kindknoten interpretiert, die durch die DTD nicht erlaubt
sind.
    \section[Entwurf einer XSD]{Entwurf einer XSD für \BibTeXXML}

Eine XML-Schema-Definition (XSD) eignet sich hervorragend, um das
\BibTeXXMLformat{} zu definieren. Es lässt sich sehr genau einschränken, welche
Werte an verschiedenen Stellen im Dokument erlaubt sind (z.B. für die ID eines
Eintrags oder den Namen eines Feldes). Auch kann im Gegensatz zur DTD definiert
werden, dass Eintrags-IDs eindeutig sein müssen.

Es wurde eine XSD im Venetian-Blind-Design nach der Beschreibung in
\cite{oracle:xsd-design-patterns} entworfen. Sie befindet sich im Abgabeordner
unter dem Dateinamen \file{BibTeX-XML.xsd}. Die Validierbarkeit des Dokuments
mit dieser XSD wurde mit dem Programm \file{xmllint} durch folgenden Aufruf
überprüft:
\begin{lstlisting}[language=bash]
$ xmllint --schema BibTeX-XML.xsd --noout Projekt_BIB_original.xml
Projekt_BIB_original.xml validates
\end{lstlisting}

Eine Version der erzeugten Datenbank mit XML-Schema-Instanz-Deklaration findet
sich in der Datei \file{Projekt\_BIB\_original\_xsd.xml}. Ob die
Instanz-Deklaration korrekt funktioniert konnte nicht überprüft werden, da
\file{xmllint} diese nicht automatisiert verarbeiten kann.
    \section[Transformation zu HTML]{Transformation zu HTML mittels XSLT}

\def\xsltemplateexpandvalue{\lstinlineXML|compute-value|}
\def\xsltemplatecollectionoverview{\lstinlineXML|collection-overview|}

Es wurde ein XSL-Stylesheet verfasst, das eine \BibTeXXMLdatabase{} zur besseren
Darstellung im Browser in ein HTML-Dokument umwandelt. Da dem Autor kein
XSLT-Prozessor zur Verfügung stand, der eine höhere XSLT-Version unterstützt,
wurde das Stylesheet für XSLT~1.1 entworfen. Es liegt im Abgabeorder unter dem
Namen \fBibTeXXMLtohtml. In den folgenden Abschnitten wird seine Funktion
beschrieben.

\subsection{Beschreibung des Ergebnisdokuments}
\label{subsec:beschreibung-des-ergebnisdokuments}

Das Ergebnisdokument enthält zwei Hauptabschnitte:
\begin{description}
    \item[Übersicht über alle Quellen:] In einer Tabelle werden alle Quellen der
        Datenbank zusammen mit ihrem jeweiligen Typ aufgelistet. Im
        Initialzustand werden dabei von jeder Quelle nur Autor und Titel
        angezeigt. Durch Aufklappen eines
        HTML-\mbox{\lstinlinehtml|details|-}Elements werden die restlichen Infos
        über die jeweilige Quelle offenbart.
        
        Ein mit CSS eingerichteter Farbcode erleichtert die Unterscheidung
        zwischen den verschiedenen Quelltypen.
    \item[Übersicht über Sammlungen:] Eine \BibTeXXMLdatabase{} kann mehrere
        Quellen definieren, die aus demselben Buch, derselben Sammlung oder
        derselben Konferenz stammen. Möglicherweise findet der Betrachter der
        Datenbank solche Zusammenhänge interessant. Deshalb sammelt das
        XSL-Stylesheet diese Beziehungen und stellt sie in einem Abschnitt im
        HTML-Dokument dar: Unter der Übeschrift jeder Sammlung sind deren
        jeweilige Einzelquellen aufgelistet.
\end{description}

\subsection{Zu überwindende Schwierigkeiten}

\begin{description}
    \item[Expansion von Werten:] In \autoref{subsec:feld-und-stringwerte} wurde
        definiert, dass Werte in einer \BibTeXXMLdatabase{} zusammengesetzt sind
        aus literalen und referenzierten Werten. Um den tatsächlichen Wert eines
        Felds zu verarbeiten, muss die XSLT das Feld daher erst expandieren --
        also die Referenzen auflösen und alle Einzelwerte verketten.
        
        Dieses Problem wird durch das benannte Template
        \xsltemplateexpandvalue{} gelöst: Durch rekursives Anwenden weiterer
        Templates wird den Referenzen so lang gefolgt, bis der Prozessor auf
        einen Wert stößt, der vollständig literal definiert ist.
    \item[Einmalige Aufführung jeder Sammlung in der Sammlungsübersicht:]
        Zur Erzeugung der Sammlungsübersicht, die in
        \autoref{subsec:beschreibung-des-ergebnisdokuments} beschrieben wurde,
        werden zuerst alle Sammlungstitel (d.h. alle
        \mbox{\lstinlineplain|booktitle|-}Felder) aus der \BibTeXXMLdatabase{}
        ausgewählt. Für jeden dieser Titel muss dann die zugehörige Liste
        erzeugt werden. Da jedoch die Liste aller Sammlungstitel manche Titel
        mehrfach enthalten kann, müssen daraus zuvor die Duplikate entfernt
        werden. Ansonsten würden manche Sammlungslisten mehrfach im
        Ergebnisdokument auftauchen. Mit XPath~2.0 wäre das kein Problem -- in
        XSLT~1.1 gibt es aber mit XPath~1.0 die nötige Funktion
        \lstinlineplain|distinct-values()| nicht.
        
        Dieses Problem wird gelöst durch das Template
        \xsltemplatecollectionoverview{}. Im XSL-Stylesheet enthält dieses
        einen umfangreichen Kommentar, der das Vorgehen genau erklärt.
    \item[Namespace-Attribute in HTML:] Manche der erzeugten HTML-Elemente
        haben \mbox{\lstinlineXML|xmlns:btxml|-}Attribute erhalten mit dem Wert
        \BibTeXXMLnamespace{}. Dieses Problem konnte nicht behoben werden.
\end{description}

\subsection{Durchführung der Transformation}

\begin{description}
    \item[Manuelle Transformation:] Mit einem einfachen XSLT-Prozessor kann das
        Stylesheet auf eine \BibTeXXMLdatabase{} angewandt werden. In diesem
        Fall wurde \file{xsltproc} verwendet, um im Abgabeordner die Datei
        \fhtmlresult{} zu erzeugen:
\begin{lstlisting}[language=bash]
$ xsltproc -o Projekt_BIB_original.html BibTeX-XML-to-html.xsl \
    Projekt_BIB_original_xsd.xml
\end{lstlisting}
        Aufgrund eines Fehlers von \file{xsltproc} besitzt das Ergebnis keine
        schöne Einrückung.
    \item[Automatische Transformation:] Der Abgabeordner enthält die Datei
        \fxmlresultwithstylesheet. In dieser ist die XSL-Datei als Stylesheet
        verlinkt. Zur Anzeige mittels eines Browsers muss entweder dessen
        Cross-Origin-Request-Einschränkung abgeschaltet werden, oder man lädt
        die Datei über HTTP von einem Server. Letzteres ist mit dem
        Python-3.x-Modul \file{http.server} sehr einfach und sicherer als die
        erste Variante. Daher wird dieser Ansatz vom Autor empfohlen. Es muss
        lediglich im Abgabeordner folgender Befehl auf der Kommandozeile
        ausgeführt werden:
\begin{lstlisting}[language=bash]
$ python3 -m http.server
Serving HTTP on 0.0.0.0 port 8000 (http://0.0.0.0:8000/) ...
\end{lstlisting}
        Unter der URL {\ttfamily http://0.0.0.0:8000/} sind dann alle Inhalte
        des Abgabeordners aufrufbar -- also auch die XML-Datei mit Stylesheet,
        die auf diesem Weg vom Browser korrekt interpretiert wird.
\end{description}

    \nocite{btxdoc}
    \nocite{github-bibtex-xml}
    \printbibliography

\end{document}
