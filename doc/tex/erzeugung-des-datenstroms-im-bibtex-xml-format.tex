\section{Erzeugung des Datenstroms im \BibTeXXMLformat}

Für dieses Projekt wurden mehrere ähnliche \mbox{\BibTeX-}Datenbanken
vorgegeben, von denen eine gewählt und in das entwickelte XML-Format konvertiert
werden soll. Dieser Abschnitt erläutert die Auswahl der konkreten Datenbank und
erklärt, wie aus dieser ein \BibTeXXMLdoc{} generiert wird.

\subsection{Eigenes Konvertierungsprogramm}

\def\temp{%
    Einzige Einschränkung: Das Konvertierungsprogramm unterstützt keine
    Präambel-Elemente. Beim Versuch, eine Datenbank mit einem solchen Element zu
    konvertieren, endet das Programm mit einem Fehler.%
}

Der Autor hat sich dafür entschieden, selbst einen \mbox{\BibTeX-}Parser in
Haskell zu implementieren. Auf dieser Basis hat er dann ein
Konvertierungsprogramm geschrieben, das aus beliebigen
\mbox{\BibTeX-}Datenbanken\footnote{\temp} ein äquivalentes valides XML-Dokument
im zuvor beschriebenen \BibTeXXMLformat{} erzeugt. Das Programm findet sich im
GitHub-Repository dieses Projekts unter dem Namen \file{bibtoxml} (siehe
\cite{github-bibtex-xml}).

\subsection{Auswahl der Datenbank}

Sowohl \BibTeXXML{} als auch \file{bibtoxml} weisen keine Einschränkungen auf,
die für die Erfüllung der Aufgabenstellung von Bedeutung sind. Die vorgegebene
Maximal-Datenbank \file{Projekt\_BIB\_original.txt} lässt sich mit
\file{bibtoxml} so umwandeln, dass das Ergebnis keine Fehler enthält. Das
Konvertierungsprogramm sortiert dabei automatisch Einträge aus, die Fehler im
Sinne der Sprache einer \BibTeXdatabase{} enthalten. Da sie kein Problem für den
Parser darstellt, wird für das Projekt die genannte Maximal-Datenbank verwendet
-- im Folgenden bezeichnet als \enquote{die \BibTeXdatabase}. Diese ist im
Abgabeordner unter dem genannten Dateinamen enthalten.

\subsection{Umwandlung der Datenbank}

\begin{flushleft}
Der folgende Befehl auf einer Linux-Kommandozeile erzeugt aus der  \BibTeXdatabase{} das \BibTeXXMLdoc{} \file{Projekt\_BIB\_original.xml}:
\end{flushleft}
\begin{lstlisting}[language=bash]
$ bibtoxml -o Projekt_BIB_original.xml -l Projekt_BIB_original.log Projekt_BIB_original.txt
\end{lstlisting}
Außerdem werden die Log-Ausgaben in die Datei \file{Projekt\_BIB\_original.log}
ausgegeben. Beide erzeugte Dateien befinden sich mit den hier verwendeten
Dateinamen im Abgabeordner.

\subsection{Ausgeschlossene Einträge}

Grundsätzlich ist es vor der Verwendung von \file{bibtoxml} nicht nötig,
Einträge aus einer umzuwandelnden Datenbank auszuschließen. Das Programm kann
erfolgreich jede \BibTeXdatabase{} verarbeiten. Allerdings schließt
\file{bibtoxml} -- wie bereits erwähnt -- selbständig fehlerhafte Daten aus. Der
Datei \file{Projekt\_BIB\_original.log} lässt sich in diesem Fall entnehmen,
dass aus der \BibTeXdatabase{} ein einzelner Eintrag aufgrund der doppelten
Verwendung der ID \lstinlineplain|Skeide:04| entfernt wurde.