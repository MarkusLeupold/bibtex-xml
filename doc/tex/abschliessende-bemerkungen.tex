\section{Abschließende Bemerkungen des Autors}

\subsection{Unterschätzung des Entwicklungsaufwandes von \file{bibtoxml}}

Im Laufe der Bearbeitung des Projekts wurde immer deutlicher, dass die
Entwicklung eines eigenen \mbox{\BibTeX-}Parsers ein sehr umfangreiches Vorhaben
ist. Es hat mich sehr viel Zeit gekostet, \file{bibtoxml} an einen Punkt zu
bringen, an dem es fehlerfrei die vorgegebene Datenbank umwandeln kann. Dadurch
waren meine zeitlichen Ressourcen nur sehr knapp, um die eigentliche Aufgabe zu
lösen. Eine vorausschauendere Zeitplanung wäre angebracht gewesen.

\subsection{Auswirkungen von Designentscheidungen}

Ich habe für die einzelnen Teile des Projektes sehr viele Designentscheidungen
getroffen. Die drei gravierendsten sind:
\begin{itemize}
    \item das Übernehmen der String-Definitionen und -Referenzen aus der
        \BibTeXdatabase{}
    \item die ausgelassene Interpretation der \mbox{\TeX-}Kontrollsequenzen.
    \item die Unterstützung beliebiger \BibTeXdatabase en mit unbegrenztem
        Komplexitätsgrad, statt einer für die vorgegebene Datenbank
        maßgeschneiderten Lösung
\end{itemize}
Alle drei Entscheidungen lassen sich damit begründen, dass ich eine möglichst
große Äquivalenz in der Aussagekraft und den Features von \BibTeX{} und
\BibTeXXML{} erzielen wollte. Zwar habe ich diese Äquivalenz weitestgehend
erreicht, aber dadurch ist es nun schwieriger, eine \BibTeXXMLdatabase{} zu
interpretieren und anzuzeigen.

So müssen Werte erst expandiert werden, bevor sie verwendbar sind, was auch das
Sortieren von Datensätzen erschwert: In der aktuellen Implementation werden
Einträge durch XSLT nach ihrer ID sortiert, weil es ohne erheblichen Mehraufwand
nicht möglich wäre, sie korrekt nach ihrem Autor oder Titel zu sortieren.

Außerdem sind jegliche syntaktischen Elemente von \TeX{} in dem erzeugten
HTML-Dokument noch nicht in normalen Text umgewandelt. Mein Plan war es, dies
der interpretierenden Anwendung zu überlassen -- so wie auch bei \BibTeX{} die
\mbox{\TeX-}Makros erst von der Anwendung, \TeX, interpretiert werden (d.h. die
\mbox{\TeX-}Syntax soll erst durch das XSL-Stylesheet oder das HTML-Dokument
unter Zuhilfenahme von JavaScript verarbeitet werden). Die Implementation eines
solchen Features ist am Ende an der nicht verfügbaren Zeit gescheitert.

\subsection{Erweiterungsmöglichkeiten}

\begin{itemize}
    \item Interpretation der \mbox{\TeX-}Syntax für die Darstellung im
        HTML-Dokument
    \item Erweiterung des CSS-Stylesheets für das HTML-Resultat zur Verbesserung
        der User-Experience, z.B.:
        \begin{itemize}
            \item Registerkarten für die verschiedenen Hauptabschnitte um
                spätere Abschnitte leichter zugänglich zu machen
            \item Makrotypographie für bessere Lesbarkeit
        \end{itemize}
    \item Einbau von Statistiken in das HTML-Resultat, z.B. ein Kreisdiagramm
        über die relativen Häufigkeiten aller Quelltypen in einer Datenbank.
    \item Sortierfunktion für die Quellenlisten (bisher nur statisch nach ID
        sortiert)
    \item Such- und Filterfunktionen
\end{itemize}