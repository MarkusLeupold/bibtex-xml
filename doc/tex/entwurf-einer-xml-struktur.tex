\section[Entwurf einer XML-Struktur]{Entwurf einer XML-Struktur zur Repräsentation einer \mbox{\BibTeX-}Datenbank}

\def\temp{
    Die Präambel ist speziell für die Verwendung mit \TeX{} gedacht. Da \TeX{}
    eine turingvollständige Sprache ist, ist die Interpretation entsprechend
    aufwendig.
}

Die XML-Struktur zur Darstellung einer \mbox{\BibTeX-}Datenbank (im Folgenden
\enquote{\BibTeXXML}) soll im Rahmen dieser Arbeit möglichst genau den
Funktionsumfang des normalen \mbox{\BibTeX-}Formats übernehmen. Neben Einträgen
sollen deshalb auch Strings unterstützt werden. Präambel-Elemente vernachlässigt
die XML-Struktur, da deren Interpretation den Rahmen dieser Arbeit sprengen
würde\footnote{\temp} und dieses Feature wahrscheinlich auch nur sehr selten
Verwendung findet.

Alle XML-Elemente des \BibTeXXMLformat s gehören dem Namensraum
\lstinlineXML|BibTeX| mit der URL {\ttfamily \BibTeXXMLnamespace} an.
\autoref{fig:structure-bibtex-xml-document} gibt einen Überblick über die
Struktur eines \BibTeXXMLdoc s. In den folgenden Unterabschnitten werden
ausgewählte \mbox{\BibTeX-}XML-Elemente und Designentscheidungen genauer
erklärt.

\begin{figure}
    \centering
    \begin{tikzpicture}
    [ with background
    , semithick
    , every node/.style={anchor=base, text depth=0}
    , node distance=\baselineskip
    ]
    \begin{scope}
        [ tree layout
        , significant sep=1em
        , record node names
        ]

        \node[xml element, dashed] (xml root) {/}
            child
              { node[xml element] {BibTeX:database}
                child
                  { node[operator] {$*$}
                    child
                      { node[operator] {$|$}
                        child
                          { node[xml element] {BibTeX:entry}
                            child {node[xml attribute] {id}}
                            child {node[xml attribute] {type}}
                            child
                              { node[operator] {$*$}
                                child
                                  { node [xml element] {BibTeX:field}
                                    child {node[xml attribute] {name}}
                                    child {node[production] {valueProd\/}}
                                  }
                              }
                          }
                        child
                          { node [xml element] {BibTeX:string}
                            child {node[xml attribute] {name}}
                            child {node[production] {valueProd\/}}
                          }
                      }
                  }
              }
          ;
    \end{scope}
    \node[fit=\recordednodenames, rectangle] (main tree) {};

    \node
      (valueProd)
      [ production
      , matrix
      , matrix anchor=north
      , below=of main tree.south
      , label=
          { [production,decorate=false,anchor=north west,draw=none]
            north west:%
            valueProd:\/
          }
      ]
      { \begin{scope}[tree layout]
            \node[operator] {$+$}
                child
                  { node[operator] {$|$}
                    child {node[xml element] {BibTeX:literalValue}}
                    child {node[xml element] {BibTeX:referencedValue}}
                  }
              ;
        \end{scope}
        \\
      };
    \node
      (symbolKey)
      [ rectangle
      , draw
      , matrix
      , matrix anchor=north
      , below=of valueProd.south
      , column sep=1em
      , empty/.style={text height=0, inner sep=0, minimum size=2ex}
      ]
      { \node[xml element, empty, label={[anchor=west]east:XML-Element}] {};
      & \node
          [ production
          , empty
          , label=
              { [anchor=west]
                east:Produktion (Platzhalter für Baumbestandteil)
              }
          ]
          {}
          ;
      \\\node
          [ xml attribute
          , empty
          , label={[anchor=west]east:XML-Attribut}
          ]
          {}
          ;
      \\\node[operator, empty, label={[anchor=west]east:Operator}] {};
      \\
      };
\end{tikzpicture}

    \caption{Struktur eines \mbox{\BibTeX-}XML-Dokuments}
    \label{fig:structure-bibtex-xml-document}
\end{figure}

\subsection{Das Datenbankelement}

Das Wurzelelement einer \mbox{\BibTeX}-XML-Datenbank ist ein Element vom Typ
\mbox{\lstinlineXML|BibTeX:database|.} Dieses enthält -- analog zur Struktur
einer \mbox{\BibTeX-}Datenbank -- in beliebiger Abfolge Eintrags- und
String-Elemente. Da \mbox{\BibTeX-}Datenbanken keine Metainformationen besitzen,
erfordert das \mbox{\lstinlineXML|BibTeX:database|-}Element außer den genannten
Elementen keine Attribute oder sonstigen Kindelemente.

\subsection{Eintrags-Elemente}

Einträge werden durch den Elementtyp \lstinlineXML|BibTeX:entry| dargestellt.
Deren Struktur entspricht im Großen und Ganzen den Einträgen in einer normalen
\mbox{\BibTeX-}Datenbank. Für die einzelnen Felder gibt es den speziellen
Elementtyp \lstinlineXML|BibTeX:field|, von dem null oder mehr Kindelemente in
jedem Eintrag vorhanden sein können. Einträge dürfen also theoretisch leer sein,
wobei das interpretierende Programm dann einen Fehler ausgeben sollte. Diese
Entscheidung rührt daher, dass auch das \mbox{\file{bibtex}-}Paket in \LaTeX{}
nicht mit einem Fehler stoppt, wenn Pflichtfelder in einem Eintrag fehlen,
sondern nur Warnungen ausgibt.

Der Wert eines Felds hat die in \autoref{subsec:feld-und-stringwerte}
beschriebene Struktur.

\subsection{Stringdefinitionen}

Während in einer \mbox{\BibTeX-}Datenbank Alias-Definitionen immer in
String-Elemente eingefasst sind, werden in \BibTeXXML{} Aliase einzeln auf
Datenbanklevel definiert. Ein umgebendes String-Element nach dem Vorbild der
\mbox{\BibTeX-}Datenbank würde zu den eigentlichen Definitionen keine weiteren
Informationen ergänzen, sodass eine solche Schachtelung nur die Dokumentstruktur
unnötigerweise komplizierter machen würde. Jeder Alias wird durch ein Element
vom Typ \lstinlineXML|BibTeX:string| definiert. Der Wert eines einzelnen Strings
hat dieselbe Struktur wie der Wert eines Felds -- beschrieben in
\autoref{subsec:feld-und-stringwerte}.

\subsection{Feld- und Stringwerte}
\label{subsec:feld-und-stringwerte}

Felder und Strings können zusammengesetzte Werte haben. Jeder Bestandteil eines
Wertes ist dabei entweder ein literaler Wert oder eine Referenz auf einen
String. Daher lassen sich Feld- und Stringwerte durch eine beliebige Abfolge
mindestens eines Elements der folgenden zwei Typen darstellen:
\begin{description}
    \item[\ttfamily BibTeX:literalValue:] Ein literaler Wert. Jedes
        Element dieses Typs besitzt ein Kind vom XML-Typ \lstinlineXML|CDATA|,
        das exakt der Zeichenkette entspricht, die das Element repräsentiert.
        Der Wert eines \mbox{\lstinlineXML|BibTeX:literalValue|-}Elements ist
        damit genau dessen Inhalt.
    \item[\ttfamily BibTeX:referencedValue:] Eine Referenz auf einen
        String. Jedes Element dieses Typs ist leer aber besitzt ein
        \mbox{\lstinlineXML|name|-}Attribut, das den Namen des referenzierten
        Strings enthält. Der Wert eines
        \mbox{\lstinlineXML|BibTeX:referencedValue|-}Elements bestimmt sich wie
        folgt:
        \begin{enumerate}
            \item Sei \textit{r} das
                \mbox{\lstinlineXML|BibTeX:referencedValue|-}Element, dessen
                Wert bestimmt werden soll.
            \item Liste alle Elemente des vollständigen Dokumentbaums in
                Präorder auf. Das Resultat nennen wir die \textit{Liste aller
                Elemente}. Fakt: Diese Liste enthält \textit{r}.
            \item Suche in der Liste aller Elemente das letzte
                \mbox{\lstinlineXML|BibTeX:string|-}Element, das vor \textit{r}
                steht und dessen \mbox{\lstinlineXML|name|-}Attribut mit dem
                \mbox{\lstinlineXML|name|-}Attribut von \textit{r}
                übereinstimmt. Wenn dieses Element existiert, so ist der Wert
                von \textit{r} der Wert dieses Elements. Ansonsten behandle das
                Fehlen als einen Fehler (\textit{r} referenziert einen String,
                der nicht definiert ist).
        \end{enumerate}
\end{description}
Der Wert eines Felds oder Strings bestimmt sich durch Verkettung der Einzelwerte
seiner Kinder in Reihenfolge der Dokumentordnung.