\section[Transformation zu HTML]{Transformation zu HTML mittels XSLT}

\def\xsltemplateexpandvalue{\lstinlineXML|compute-value|}
\def\xsltemplatecollectionoverview{\lstinlineXML|collection-overview|}

Es wurde ein XSL-Stylesheet verfasst, das eine \BibTeXXMLdatabase{} zur besseren
Darstellung im Browser in ein HTML-Dokument umwandelt. Da dem Autor kein
XSLT-Prozessor zur Verfügung stand, der eine höhere XSLT-Version unterstützt,
wurde das Stylesheet für XSLT~1.1 entworfen. Es liegt im Abgabeorder unter dem
Namen \fBibTeXXMLtohtml. In den folgenden Abschnitten wird seine Funktion
beschrieben.

\subsection{Beschreibung des Ergebnisdokuments}
\label{subsec:beschreibung-des-ergebnisdokuments}

Das Ergebnisdokument enthält zwei Hauptabschnitte:
\begin{description}
    \item[Übersicht über alle Quellen:] In einer Tabelle werden alle Quellen der
        Datenbank zusammen mit ihrem jeweiligen Typ aufgelistet. Im
        Initialzustand werden dabei von jeder Quelle nur Autor und Titel
        angezeigt. Durch Aufklappen eines
        HTML-\mbox{\lstinlinehtml|details|-}Elements werden die restlichen Infos
        über die jeweilige Quelle offenbart.
        
        Ein mit CSS eingerichteter Farbcode erleichtert die Unterscheidung
        zwischen den verschiedenen Quelltypen.
    \item[Übersicht über Sammlungen:] Eine \BibTeXXMLdatabase{} kann mehrere
        Quellen definieren, die aus demselben Buch, derselben Sammlung oder
        derselben Konferenz stammen. Möglicherweise findet der Betrachter der
        Datenbank solche Zusammenhänge interessant. Deshalb sammelt das
        XSL-Stylesheet diese Beziehungen und stellt sie in einem Abschnitt im
        HTML-Dokument dar: Unter der Übeschrift jeder Sammlung sind deren
        jeweilige Einzelquellen aufgelistet.
\end{description}

\subsection{Zu überwindende Schwierigkeiten}

\begin{description}
    \item[Expansion von Werten:] In \autoref{subsec:feld-und-stringwerte} wurde
        definiert, dass Werte in einer \BibTeXXMLdatabase{} zusammengesetzt sind
        aus literalen und referenzierten Werten. Um den tatsächlichen Wert eines
        Felds zu verarbeiten, muss die XSLT das Feld daher erst expandieren --
        also die Referenzen auflösen und alle Einzelwerte verketten.
        
        Dieses Problem wird durch das benannte Template
        \xsltemplateexpandvalue{} gelöst: Durch rekursives Anwenden weiterer
        Templates wird den Referenzen so lang gefolgt, bis der Prozessor auf
        einen Wert stößt, der vollständig literal definiert ist.
    \item[Einmalige Aufführung jeder Sammlung in der Sammlungsübersicht:]
        Zur Erzeugung der Sammlungsübersicht, die in
        \autoref{subsec:beschreibung-des-ergebnisdokuments} beschrieben wurde,
        werden zuerst alle Sammlungstitel (d.h. alle
        \mbox{\lstinlineplain|booktitle|-}Felder) aus der \BibTeXXMLdatabase{}
        ausgewählt. Für jeden dieser Titel muss dann die zugehörige Liste
        erzeugt werden. Da jedoch die Liste aller Sammlungstitel manche Titel
        mehrfach enthalten kann, müssen daraus zuvor die Duplikate entfernt
        werden. Ansonsten würden manche Sammlungslisten mehrfach im
        Ergebnisdokument auftauchen. Mit XPath~2.0 wäre das kein Problem -- in
        XSLT~1.1 gibt es aber mit XPath~1.0 die nötige Funktion
        \lstinlineplain|distinct-values()| nicht.
        
        Dieses Problem wird gelöst durch das Template
        \xsltemplatecollectionoverview{}. Im XSL-Stylesheet enthält dieses
        einen umfangreichen Kommentar, der das Vorgehen genau erklärt.
    \item[Namespace-Attribute in HTML:] Manche der erzeugten HTML-Elemente
        haben \mbox{\lstinlineXML|xmlns:btxml|-}Attribute erhalten mit dem Wert
        \BibTeXXMLnamespace{}. Dieses Problem konnte nicht behoben werden.
\end{description}

\subsection{Durchführung der Transformation}

\begin{description}
    \item[Manuelle Transformation:] Mit einem einfachen XSLT-Prozessor kann das
        Stylesheet auf eine \BibTeXXMLdatabase{} angewandt werden. In diesem
        Fall wurde \file{xsltproc} verwendet, um im Abgabeordner die Datei
        \fhtmlresult{} zu erzeugen:
\begin{lstlisting}[language=bash]
$ xsltproc -o Projekt_BIB_original.html BibTeX-XML-to-html.xsl \
    Projekt_BIB_original_xsd.xml
\end{lstlisting}
        Aufgrund eines Fehlers von \file{xsltproc} besitzt das Ergebnis keine
        schöne Einrückung.
    \item[Automatische Transformation:] Der Abgabeordner enthält die Datei
        \fxmlresultwithstylesheet. In dieser ist die XSL-Datei als Stylesheet
        verlinkt. Zur Anzeige mittels eines Browsers muss entweder dessen
        Cross-Origin-Request-Einschränkung abgeschaltet werden, oder man lädt
        die Datei über HTTP von einem Server. Letzteres ist mit dem
        Python-3.x-Modul \file{http.server} sehr einfach und sicherer als die
        erste Variante. Daher wird dieser Ansatz vom Autor empfohlen. Es muss
        lediglich im Abgabeordner folgender Befehl auf der Kommandozeile
        ausgeführt werden:
\begin{lstlisting}[language=bash]
$ python3 -m http.server
Serving HTTP on 0.0.0.0 port 8000 (http://0.0.0.0:8000/) ...
\end{lstlisting}
        Unter der URL {\ttfamily http://0.0.0.0:8000/} sind dann alle Inhalte
        des Abgabeordners aufrufbar -- also auch die XML-Datei mit Stylesheet,
        die auf diesem Weg vom Browser korrekt interpretiert wird.
\end{description}